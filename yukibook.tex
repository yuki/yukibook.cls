\documentclass{yukibook}

\begin{document}

\yukibook{The Title \linebreak of the Book} 	% Title
  {Rubén Gómez}  % Author
  {2021-2022}    % Year
  {The name \linebreak of the degree} % Name of degree
  {\textquote{A very good phrase for a very good book}}	% catch phrase
  {The phrase's author}	% the phrase's author
  {img/cover.png}

%--------------------------------------------------------------------------
% Start your parts, chapters and sections here
%--------------------------------------------------------------------------

%\part{Part 1}

\chapter{Introduction}
You can see how this pdf has been made \href{https://github.com/yuki/yukibook.cls}{here} using a custom \LaTeX{}  class.

You can use the \textbf{yukibook.tex} file to see the text and the file \textbf{yukibook.cls} to see how the class has been implemented. If you don't know nothing about \LaTeX{} , probably you won't need this.



\begin{tcolorbox}
This is a basic tcolorbox.

\Blindtext[1]
\end{tcolorbox}

\begin{tcolorbox}[title=Tcolorbox with title]
\Blindtext[1]
\end{tcolorbox}

The next box is a fitted text into a 4cm heigth box:


Txo tcboxfit in a row with fitted box:
\begin{tcbraster}[colback=green!10!white,boxsep=1mm]
\tcboxfit[height=4cm]{\Blindtext[1]}
\tcboxfit[height=4cm]{\Blindtext[1]}
\end{tcbraster}

%\tcbset{colframe=blue!50!black,colback=red!10!white,
%boxsep=0pt,top=1mm,bottom=1mm,left=1mm,right=1mm,
%fit algorithm=hybrid*,raster equal skip=1mm}


\Blindtext[1]

\chapter{Chapter 2}
\Blindtext[1]

\section{Nueva sección 1}
\Blindtext[1]

\subsection{nueva subsección}
\Blindtext[1]

\subsubsection{sub sub sección}
\Blindtext[1]

\subparagraph{sub párrafo}
\Blindtext[1]

\section{Nueva sección 2: Awesomebox}
Prueba de notebox:

\notebox{\Blindtext[1]}

Prueba de tipbox:
\tipbox{\Blindtext[1]}

Prueba de warningbox:
\warningbox{\Blindtext[1]}

Prueba de cautionbox:
\cautionbox{\Blindtext[1]}

Prueba de importantbox:
\importantbox{\Blindtext[1]}

\awesomebox{5pt}{\faCogs}{black}{\Blindtext[1]}

\begin{info-box}
  \notebox{\Blindtext[1]}
\end{info-box}
\begin{warning-box}
  \Blindtext[1]
\end{warning-box}
\begin{error-box}
  \Blindtext[1]
\end{error-box}


\part{Part 2}
\Blinddocument


\end{document}