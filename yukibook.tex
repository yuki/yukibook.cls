\newcommand{\ClassPath}{.}
\documentclass{\ClassPath/yukibook}

\begin{document}

\yukibook{The Title \linebreak of the Book} % Title
  {Rubén Gómez Olivencia}  % Author
  {2021-2022}    % Year
  {The name \linebreak of the degree} % Name of degree
  {\textquote{A very good phrase for a very good book}} % catch phrase
  {The phrase's author} % the phrase's author
  {img/cover.png}
  {28436c} % color in HTML format

\coverpage
\licensepage

\tableofcontents


%--------------------------------------------------------------------------
% Start your parts, chapters and sections here
%--------------------------------------------------------------------------

\part{Part 1}

\chapter{Introduction}
You can see how this pdf has been made \href{https://github.com/yuki/yukibook.cls}{here} using a custom \LaTeX{}  class.

You can use the \textbf{yukibook.tex} file to see how to structure the text and the file \textbf{yukibook.cls} to see how the class has been implemented. If you don't know nothing about \LaTeX{} , probably you won't need this.

In this example file I'm going to show all the custom features I'm going to use in my books.

\chapter{Required Packages}
To use this class there are some requirements that must be working in order to recreate the PDF you are reading.
\section{\LaTeX{} packages}

This class uses some packages that should be installed with your \LaTeX{} distribution.

You can see all needed packages searching for “RequirePackage” in the \textbf{yukibook.cls} file.



\section{Command line packages}
The \LaTeX{} package 'minted' needs the \href{https://pygments.org/}{pygmentize} CLI program to work. Be sure to have it installed before using the class.


\chapter{Examples}
This class has been done to use it to create technichal books for students. In those books will appear source code, tips, warnings, ...

\section{Custom code boxes}
I have created a custom box, using the \textbf{tcolorbox} package, called \textbf{mycode}.

For example, the next code for GNU/Linux console:

\begin{mycode}{Journald configuration file}{console}{}
[root@localhost ~]# vi /etc/systemd/journald.conf
\end{mycode}

There are a lot of possible output examples that can be made:

\begin{mycode}{lsblk command output}{shell-session}{}
[root@localhost ~]# lsblk
NAME                       MAJ:MIN RM   SIZE RO TYPE MOUNTPOINTS
nvme0n1                    259:0    0 931,5G  0 disk
├─nvme0n1p1                259:1    0   512M  0 part
└─nvme0n1p2                259:2    0   800G  0 part /home
nvme1n1                    259:3    0 931,5G  0 disk
├─nvme1n1p1                259:4    0   512M  0 part /boot/efi
└─nvme1n1p2                259:5    0    90G  0 part /
\end{mycode}


Another example with MySQL output text, showing some replica status:

\begin{mycode}{Mysql: show replicas}{mysql}{\footnotesize}
mysql> show replicas;
+-----------+---------+------+-----------+--------------------------------------+
| Server_Id | Host    | Port | Source_Id | Replica_UUID                         |
+-----------+---------+------+-----------+--------------------------------------+
|         2 | server2 | 3306 |         1 | 9fb33895-4c78-11ec-b23b-525400f50be5 |
+-----------+---------+------+-----------+--------------------------------------+
1 row in set (0,00 sec)
\end{mycode}

Sometimes we need inline console code like  \commandbox{ls} , other times we need to talk about some configuration file like \configfile{/etc/hosts}, config directory \configdir{/etc}  inline text \inlineconsole{help} or a movie \movie{https://www.imdb.com/title/tt2084970/}{The Imitation Game}



\section{Custom boxes}
Se han creado unas cajas especiales para resaltar ciertos apartados. Lo siguiente es una caja de información (\textbf{infobox}):

\infobox{Esto es información que nos viene bien}

Este otro ejemplo es de un \textbf{warningbox}:
\warnbox{Hay que tener cuidado con lo que aparece aquí}

Prueba de \textbf{errorbox}:
\errorbox{Este texto hay que tenerlo muy en cuenta}

Prueba de \textbf{questionbox}:
\questionbox{Este texto es para hacer una pregunta}

Prueba de \textbf{exercisebox}:
\exercisebox{Este texto es para hacer un ejercicio}


\section{Custom tables}

Custom tables created using the package \textbf{tabularray}, and the custom table name is \textbf{yukitblr}:

\begin{yukitblr}{XXX}
    Head & Head & Head \\
    Alpha & Beta & Gamma \\
    Delta & Epsilon & Zeta  \\
    Eta & Theta & Iota \\
    Kappa & Lambda & Mu \\
    Nu & Xi & Omicron \\
    Pi & Rho & Sigma \\
    Tau & Upsilon & Phi \\
\end{yukitblr}

Another custom table:

\begin{yukitblrcol}{XXX}
     & Head & Head \\
    Advantages & Beta & Gamma \\
    Drawbacks & Epsilon & Zeta  \\
\end{yukitblrcol}

Another table with more text in it. To use lists inside tables it must add “measure=vbox” to the table

\begin{yukitblr}{colspec={X[j]X},measure=vbox}
    Head & Head \\
    one & two   \\
    \blindtext[1] & two  \\
    \begin{itemize}
        \item qwe
        \item qwe
        \item qwe
        \item qwe
    \end{itemize} & two  \\
\end{yukitblr}



\part{Part 2}
\Blinddocument

%\tcblistof{mycommands}{My Listings}
%\listoflistings

\end{document}