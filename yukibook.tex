\documentclass{yukibook}


\title{Título}
\author{Rubén Gómez}



\begin{document}

\maketitle
\tableofcontents

\part{Nueva parte 1}

\chapter{Nuevo capítulo 1}
You can see how this pdf has been made \href{https://github.com/yuki/yukibook.cls}{here} or you can go to the next url: \url{https://github.com/yuki/yukibook.cls} .


\Blindtext[2]

\chapter{Nuevo capítulo 2}
\Blindtext[2]

\section{Nueva sección 1}
\Blindtext[2]

\subsection{nueva subsección}
\Blindtext[1]

\subsubsection{sub sub sección}
\Blindtext[1]

\subparagraph{sub párrafo}
\Blindtext[1]

\section{Nueva sección 2: Awesomebox}
Prueba de notebox:
\notebox{\Blindtext[1]}

Prueba de tipbox:
\tipbox{\Blindtext[1]}

Prueba de warningbox:
\warningbox{\Blindtext[1]}

Prueba de cautionbox:
\cautionbox{\Blindtext[1]}

Prueba de importantbox:
\importantbox{\Blindtext[1]}

\awesomebox{5pt}{\faCogs}{black}{\Blindtext[1]}

\begin{info-box}
  \notebox{\Blindtext[1]}
\end{info-box}
\begin{warning-box}
  \Blindtext[1]
\end{warning-box}
\begin{error-box}
  \Blindtext[1]
\end{error-box}

\Blinddocument


\end{document}