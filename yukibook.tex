\documentclass{yukibook}

\begin{document}

\yukibook{The Title \linebreak of the Book} 	% Title
  {Rubén Gómez}  % Author
  {2021-2022}    % Year
  {The name \linebreak of the degree} % Name of degree
  {\textquote{A very good phrase for a very good book}}	% catch phrase
  {The phrase's author}	% the phrase's author
  {img/cover.png}

%--------------------------------------------------------------------------
% Start your parts, chapters and sections here
%--------------------------------------------------------------------------

%\part{Part 1}

\chapter{Introduction}
You can see how this pdf has been made \href{https://github.com/yuki/yukibook.cls}{here} using a custom \LaTeX{}  class.

You can use the \textbf{yukibook.tex} file to see how to structure the text and the file \textbf{yukibook.cls} to see how the class has been implemented. If you don't know nothing about \LaTeX{} , probably you won't need this.

In this example file I'm going to show all the custom features I'm going to use in my books. 

\chapter{Required Packages}
To use this class there are some requirements that must be working in order to recreate the PDF you are reading.
\section{\LaTeX{} packages}

This class uses the next \LaTeX{} packages:
\begin{itemize}
  \item babel
  \item geometry
  \item sourcesanspro
  \item sourcecodepro
  \item xcolor
  \item fancyhdr
  \item tikz
  \item titlesec
  \item hyperref
  \item tcolorbox
  \item minted 
  \item awesomebox
\end{itemize}

\section{Command line packages}
The \LaTeX{} package 'minted' needs the \href{https://pygments.org/}{pygmentize} CLI program to work.


\chapter{Utilities}
This class has been done to use it to create technichal books for students where's will appear source code, tips, warnings, ... 

For example, the next code for GNU/Linux console:

\begin{mycode}{Journald configuration file}{console}{}
[root@localhost ~]# vi /etc/systemd/journald.conf
\end{mycode}

There are a lot of possible output examples that can be made:

\begin{mycode}{lsblk command output}{shell-session}{}
[root@localhost ~]# lsblk
NAME                       MAJ:MIN RM   SIZE RO TYPE MOUNTPOINTS
sda                          8:0    0   1,8T  0 disk 
└─sda1                       8:1    0   1,8T  0 part /home/backup
sdb                          8:16   0   3,6T  0 disk 
└─sdb1                       8:17   0   3,6T  0 part /home/disco4tb
sdc                          8:32   0 447,1G  0 disk 
├─sdc1                       8:33   0   529M  0 part 
├─sdc2                       8:34   0   100M  0 part 
├─sdc3                       8:35   0    16M  0 part 
└─sdc4                       8:36   0 446,5G  0 part 
nvme0n1                    259:0    0 931,5G  0 disk 
├─nvme0n1p1                259:1    0   512M  0 part 
└─nvme0n1p2                259:2    0   800G  0 part /home
nvme1n1                    259:3    0 931,5G  0 disk 
├─nvme1n1p1                259:4    0   512M  0 part /boot/efi
├─nvme1n1p2                259:5    0    90G  0 part /
├─nvme1n1p3                259:6    0   300G  0 part 
│ ├─VMs-ubuntu--20.04--so1 254:0    0    10G  0 lvm  
│ ├─VMs-OS--X--Monterey    254:1    0   100G  0 lvm  
│ ├─VMs-archlinux          254:2    0    20G  0 lvm  
│ ├─VMs-manjaro--kde       254:3    0    25G  0 lvm  
│ └─VMs-manjaro--gnome     254:4    0    25G  0 lvm  
└─nvme1n1p4                259:7    0 156,2G  0 part
\end{mycode}


Another example with MySQL output text, showing some replica status:

\begin{mycode}{Mysql: show replicas}{mysql}{\footnotesize}
mysql> show replicas;
+-----------+---------+------+-----------+--------------------------------------+
| Server_Id | Host    | Port | Source_Id | Replica_UUID                         |
+-----------+---------+------+-----------+--------------------------------------+
|         2 | server2 | 3306 |         1 | 9fb33895-4c78-11ec-b23b-525400f50be5 |
+-----------+---------+------+-----------+--------------------------------------+
1 row in set (0,00 sec)
\end{mycode}

And of course, sometimes we need inline console code. Maybe to point some configuration file like \\ \mintinline{console}{/etc/vim/vimrc} or sometimes when you need to show some command line code like \mintinline{console}{ mysql> }, as an example.


\section{Nueva sección 2: Awesomebox}
Prueba de notebox:

\notebox{\Blindtext[1]}

Prueba de tipbox:
\tipbox{\Blindtext[1]}

Prueba de warningbox:
\warningbox{\Blindtext[1]}

Prueba de cautionbox:
\cautionbox{\Blindtext[1]}

Prueba de importantbox:
\importantbox{\Blindtext[1]}

\awesomebox{5pt}{\faCogs}{black}{\Blindtext[1]}

\begin{info-box}
  \notebox{\Blindtext[1]}
\end{info-box}
\begin{warning-box}
  \Blindtext[1]
\end{warning-box}
\begin{error-box}
  \Blindtext[1]
\end{error-box}


\part{Part 2}
\Blinddocument

%\tcblistof{mycommands}{My Listings}
%\listoflistings

\end{document}